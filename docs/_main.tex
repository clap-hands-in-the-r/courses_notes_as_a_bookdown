% Options for packages loaded elsewhere
\PassOptionsToPackage{unicode}{hyperref}
\PassOptionsToPackage{hyphens}{url}
%
\documentclass[
]{book}
\usepackage{amsmath,amssymb}
\usepackage{lmodern}
\usepackage{iftex}
\ifPDFTeX
  \usepackage[T1]{fontenc}
  \usepackage[utf8]{inputenc}
  \usepackage{textcomp} % provide euro and other symbols
\else % if luatex or xetex
  \usepackage{unicode-math}
  \defaultfontfeatures{Scale=MatchLowercase}
  \defaultfontfeatures[\rmfamily]{Ligatures=TeX,Scale=1}
\fi
% Use upquote if available, for straight quotes in verbatim environments
\IfFileExists{upquote.sty}{\usepackage{upquote}}{}
\IfFileExists{microtype.sty}{% use microtype if available
  \usepackage[]{microtype}
  \UseMicrotypeSet[protrusion]{basicmath} % disable protrusion for tt fonts
}{}
\makeatletter
\@ifundefined{KOMAClassName}{% if non-KOMA class
  \IfFileExists{parskip.sty}{%
    \usepackage{parskip}
  }{% else
    \setlength{\parindent}{0pt}
    \setlength{\parskip}{6pt plus 2pt minus 1pt}}
}{% if KOMA class
  \KOMAoptions{parskip=half}}
\makeatother
\usepackage{xcolor}
\usepackage{color}
\usepackage{fancyvrb}
\newcommand{\VerbBar}{|}
\newcommand{\VERB}{\Verb[commandchars=\\\{\}]}
\DefineVerbatimEnvironment{Highlighting}{Verbatim}{commandchars=\\\{\}}
% Add ',fontsize=\small' for more characters per line
\usepackage{framed}
\definecolor{shadecolor}{RGB}{248,248,248}
\newenvironment{Shaded}{\begin{snugshade}}{\end{snugshade}}
\newcommand{\AlertTok}[1]{\textcolor[rgb]{0.94,0.16,0.16}{#1}}
\newcommand{\AnnotationTok}[1]{\textcolor[rgb]{0.56,0.35,0.01}{\textbf{\textit{#1}}}}
\newcommand{\AttributeTok}[1]{\textcolor[rgb]{0.77,0.63,0.00}{#1}}
\newcommand{\BaseNTok}[1]{\textcolor[rgb]{0.00,0.00,0.81}{#1}}
\newcommand{\BuiltInTok}[1]{#1}
\newcommand{\CharTok}[1]{\textcolor[rgb]{0.31,0.60,0.02}{#1}}
\newcommand{\CommentTok}[1]{\textcolor[rgb]{0.56,0.35,0.01}{\textit{#1}}}
\newcommand{\CommentVarTok}[1]{\textcolor[rgb]{0.56,0.35,0.01}{\textbf{\textit{#1}}}}
\newcommand{\ConstantTok}[1]{\textcolor[rgb]{0.00,0.00,0.00}{#1}}
\newcommand{\ControlFlowTok}[1]{\textcolor[rgb]{0.13,0.29,0.53}{\textbf{#1}}}
\newcommand{\DataTypeTok}[1]{\textcolor[rgb]{0.13,0.29,0.53}{#1}}
\newcommand{\DecValTok}[1]{\textcolor[rgb]{0.00,0.00,0.81}{#1}}
\newcommand{\DocumentationTok}[1]{\textcolor[rgb]{0.56,0.35,0.01}{\textbf{\textit{#1}}}}
\newcommand{\ErrorTok}[1]{\textcolor[rgb]{0.64,0.00,0.00}{\textbf{#1}}}
\newcommand{\ExtensionTok}[1]{#1}
\newcommand{\FloatTok}[1]{\textcolor[rgb]{0.00,0.00,0.81}{#1}}
\newcommand{\FunctionTok}[1]{\textcolor[rgb]{0.00,0.00,0.00}{#1}}
\newcommand{\ImportTok}[1]{#1}
\newcommand{\InformationTok}[1]{\textcolor[rgb]{0.56,0.35,0.01}{\textbf{\textit{#1}}}}
\newcommand{\KeywordTok}[1]{\textcolor[rgb]{0.13,0.29,0.53}{\textbf{#1}}}
\newcommand{\NormalTok}[1]{#1}
\newcommand{\OperatorTok}[1]{\textcolor[rgb]{0.81,0.36,0.00}{\textbf{#1}}}
\newcommand{\OtherTok}[1]{\textcolor[rgb]{0.56,0.35,0.01}{#1}}
\newcommand{\PreprocessorTok}[1]{\textcolor[rgb]{0.56,0.35,0.01}{\textit{#1}}}
\newcommand{\RegionMarkerTok}[1]{#1}
\newcommand{\SpecialCharTok}[1]{\textcolor[rgb]{0.00,0.00,0.00}{#1}}
\newcommand{\SpecialStringTok}[1]{\textcolor[rgb]{0.31,0.60,0.02}{#1}}
\newcommand{\StringTok}[1]{\textcolor[rgb]{0.31,0.60,0.02}{#1}}
\newcommand{\VariableTok}[1]{\textcolor[rgb]{0.00,0.00,0.00}{#1}}
\newcommand{\VerbatimStringTok}[1]{\textcolor[rgb]{0.31,0.60,0.02}{#1}}
\newcommand{\WarningTok}[1]{\textcolor[rgb]{0.56,0.35,0.01}{\textbf{\textit{#1}}}}
\usepackage{longtable,booktabs,array}
\usepackage{calc} % for calculating minipage widths
% Correct order of tables after \paragraph or \subparagraph
\usepackage{etoolbox}
\makeatletter
\patchcmd\longtable{\par}{\if@noskipsec\mbox{}\fi\par}{}{}
\makeatother
% Allow footnotes in longtable head/foot
\IfFileExists{footnotehyper.sty}{\usepackage{footnotehyper}}{\usepackage{footnote}}
\makesavenoteenv{longtable}
\usepackage{graphicx}
\makeatletter
\def\maxwidth{\ifdim\Gin@nat@width>\linewidth\linewidth\else\Gin@nat@width\fi}
\def\maxheight{\ifdim\Gin@nat@height>\textheight\textheight\else\Gin@nat@height\fi}
\makeatother
% Scale images if necessary, so that they will not overflow the page
% margins by default, and it is still possible to overwrite the defaults
% using explicit options in \includegraphics[width, height, ...]{}
\setkeys{Gin}{width=\maxwidth,height=\maxheight,keepaspectratio}
% Set default figure placement to htbp
\makeatletter
\def\fps@figure{htbp}
\makeatother
\setlength{\emergencystretch}{3em} % prevent overfull lines
\providecommand{\tightlist}{%
  \setlength{\itemsep}{0pt}\setlength{\parskip}{0pt}}
\setcounter{secnumdepth}{5}
\usepackage{booktabs}
\ifLuaTeX
  \usepackage{selnolig}  % disable illegal ligatures
\fi
\usepackage[]{natbib}
\bibliographystyle{plainnat}
\IfFileExists{bookmark.sty}{\usepackage{bookmark}}{\usepackage{hyperref}}
\IfFileExists{xurl.sty}{\usepackage{xurl}}{} % add URL line breaks if available
\urlstyle{same} % disable monospaced font for URLs
\hypersetup{
  pdftitle={A Minimal Book Example},
  pdfauthor={John Doe},
  hidelinks,
  pdfcreator={LaTeX via pandoc}}

\title{A Minimal Book Example}
\author{John Doe}
\date{2023-09-18}

\begin{document}
\maketitle

{
\setcounter{tocdepth}{1}
\tableofcontents
}
\hypertarget{about}{%
\chapter{About}\label{about}}

This is a \emph{sample} book written in \textbf{Markdown}. You can use anything that Pandoc's Markdown supports; for example, a math equation \(a^2 + b^2 = c^2\).

\hypertarget{usage}{%
\section{Usage}\label{usage}}

Each \textbf{bookdown} chapter is an .Rmd file, and each .Rmd file can contain one (and only one) chapter. A chapter \emph{must} start with a first-level heading: \texttt{\#\ A\ good\ chapter}, and can contain one (and only one) first-level heading.

Use second-level and higher headings within chapters like: \texttt{\#\#\ A\ short\ section} or \texttt{\#\#\#\ An\ even\ shorter\ section}.

The \texttt{index.Rmd} file is required, and is also your first book chapter. It will be the homepage when you render the book.

\hypertarget{render-book}{%
\section{Render book}\label{render-book}}

You can render the HTML version of this example book without changing anything:

\begin{enumerate}
\def\labelenumi{\arabic{enumi}.}
\item
  Find the \textbf{Build} pane in the RStudio IDE, and
\item
  Click on \textbf{Build Book}, then select your output format, or select ``All formats'' if you'd like to use multiple formats from the same book source files.
\end{enumerate}

Or build the book from the R console:

\begin{Shaded}
\begin{Highlighting}[]
\NormalTok{bookdown}\SpecialCharTok{::}\FunctionTok{render\_book}\NormalTok{()}
\end{Highlighting}
\end{Shaded}

To render this example to PDF as a \texttt{bookdown::pdf\_book}, you'll need to install XeLaTeX. You are recommended to install TinyTeX (which includes XeLaTeX): \url{https://yihui.org/tinytex/}.

\hypertarget{preview-book}{%
\section{Preview book}\label{preview-book}}

As you work, you may start a local server to live preview this HTML book. This preview will update as you edit the book when you save individual .Rmd files. You can start the server in a work session by using the RStudio add-in ``Preview book'', or from the R console:

\begin{Shaded}
\begin{Highlighting}[]
\NormalTok{bookdown}\SpecialCharTok{::}\FunctionTok{serve\_book}\NormalTok{()}
\end{Highlighting}
\end{Shaded}

\hypertarget{learning-r}{%
\chapter{Learning R}\label{learning-r}}

This is notes about learning R.

Please build this simple boook!!!!! ahhhhhhhhhhhhha

\hypertarget{environment-configuration}{%
\section{Environment configuration}\label{environment-configuration}}

\begin{itemize}
\item
  install a package: install.packages(``tidyverse'')
  Once a package installed you don't need to install it again
\item
  load a library
  Once the package is installed, you need to ``load'' it in your current environment
  libary(tidyverse)
  It was a think that was disturbing me when starting R/
  install.packages takes a double quote surrounding the package name\\
  whereas library doesn't take double quote around the package name
\item
  options(digits =3) \# 3 significant digits
\item
  installing two or more packages at once with c():\\
  install.packages(c(``tidytext'',``gutenbergr''))
\item
  question mark + name of the function to get the help of the function\\
  It opens in the bottom right panel in most configurations of R Studio.
  example : ?gutenberg\_metadata()
\item
  get the list of files in the current working directory
  list.files()
\item
  get the path of the working directory
  getwd()
\item
  set the path to the directory you want to be the working dir:
  setwd(``C:/users/etc/'')
  in Windows, files are displayed with backslashes \textbackslash{} whereas to indicate
  a file to R, you need to replace backslashes with slashes
\item
  taper simplement \textbf{library()} :\\
  nous donne la liste de nos packages disponibles
\end{itemize}

\hypertarget{basic-manipulations}{%
\section{Basic manipulations}\label{basic-manipulations}}

\begin{itemize}
\item
  remove all objects from current environment:\\
  rm(list=ls())
\item
  create a vector with rep (repeat)
\end{itemize}

my\_vec = rep(c(1,2), times = 7)
or

my\_vec2 = rep(c(1,2), times=c(3,8))

\begin{itemize}
\tightlist
\item
  print and cat
\end{itemize}

\begin{quote}
cat(`hello')
hello
print(`hello')
{[}1{]} ``hello''
\end{quote}

\begin{itemize}
\item
  Display a field with two or three\ldots{} digits:\\
  format(x, nsmall =2)
  example
  df \%\textgreater\% group\_by(mois) \%\textgreater\% summarise(format(sum(resultats),nsmall=2))
\item
  save a data frame in Rda format :\\
  save(my\_df, file = ``df\_xxx\_xxx.Rda'')
\item
  load a data frame you previously saved in Rda format :\\
  load(file = ``df\_xxx\_xxx.Rda'')
\item
  see built-ins datasets t :\\
  ls(``package:datasets'')
\item
  give name to each field of a vector or dataframe:
  setNames(vec,c(`titi',`toto',`tata'))
  setNames(df,c(`titi',`toto',`tata'))
\item
  pull()
  pull() is similar to \$ . It's mostly useful because it looks a little nicer in pipes, it also works with remote data frames, and it can optionally name the output.
  data \textless- data.frame(x1 = 1:5, x1=LETTERS{[}1:5{]})
  data \textless- setNames(data,c(``c1'',``c2''))
\end{itemize}

pull(data,c1) \# apply function with column name (extraire le vecteur c1 du dataframe)
pull(data,c2)

pull(data,1) \# apply function with index

pull returns a vector

\hypertarget{more-advanced-manipulations}{%
\section{More advanced manipulations}\label{more-advanced-manipulations}}

\begin{itemize}
\item
  Replace Na's:\\
\item
  one first solution :\\
  df \%\textgreater\% filter(code\_rub \%in\% rub\_departs) \%\textgreater\%
  group\_by(code\_rub,RUB\_AXE\_ANA\_ME,annee) \%\textgreater\% summarise(mtt = sum(resultats)) \%\textgreater\%
  pivot\_wider(names\_from=annee,values\_from=mtt) \%\textgreater\% \textbf{mutate\_if(is.numeric, \textasciitilde replace\_na(.,0)) \%\textgreater\%}
\item
  one second solution :
  df \%\textgreater\% \textbf{mutate\_all(.,\textasciitilde replace(.,is.na(.),0))}
\item
  Add totals at the bottom of a data frame :\\
  this is one of the most painful task in R, whereas it seems completely obvious in R for example.
  why? cause, dataframes are mostly data structures. They are not done to display datas.
  however, package janitor helps doing this task very easily (thank you!)
  library(janitor)
  df \%\textgreater\% adorn\_totals(fill='' ``)
  fill arguments allow to chose what will be displayed when the column is not numbers
\end{itemize}

\hypertarget{how-to-explore-a-data-frame}{%
\section{How to explore a data frame?}\label{how-to-explore-a-data-frame}}

\begin{itemize}
\item
  check the type of a field with \textbf{class}\\
  class(df\$mois) -\textgreater{} date
\item
  df \%\textgreater\% head()
\item
  df \%\textgreater\% tail()
\end{itemize}

In both cases, you can specify the number of rows at the top or at the bottom you want to see.\\
For example head(3) or tail(3)

Unique values of a field. Two techniques :
* df \%\textgreater\% distinct(mois)
* unique(df\$mois)
The two solutions give the same results. The only difference is that the first solution returns a data frame whereas the second returns a vector.

\begin{itemize}
\tightlist
\item
  Display all rows from a dataframe in dplyr:\\
  with pipe print \%\textgreater\% print(n=nrow(.))
  df \%\textgreater\% filter(mois==`2023-06-01') \%\textgreater\% group\_by(code\_rub) \%\textgreater\% summarise(format(sum(resultats),nsmall=2)) \%\textgreater\% print(n = nrow(.))
\end{itemize}

\hypertarget{about-stringr-package}{%
\section{About stringr package}\label{about-stringr-package}}

\textbf{Stringr} is more coherent than base R functions for strings treatments.\\
Stringr functions always begin with prefix \textbf{str\_} ; the first argument is always the string you want to treat. And then comes the pattern you want to identify.

Most common and useful functions in Stringr :

\begin{itemize}
\tightlist
\item
  str\_detect() -\textgreater{} returns a logical vector (a vector of TRUE and FALSE)
\item
  str\_subset()
\item
  str\_view()
\item
  str\_view\_all()
\item
  str\_replace()
\item
  str\_replace\_all()
\item
  str\_split()
\item
  str\_trim()
\item
  str\_to\_lower()
\end{itemize}

\hypertarget{about-regex-in-r}{%
\section{About Regex in R}\label{about-regex-in-r}}

\hypertarget{special-characters}{%
\subsection{Special characters}\label{special-characters}}

\begin{itemize}
\tightlist
\item
  \textbackslash\textbackslash d stands for \textbf{one of any digit 0,1,2, up to 9}
\item
  \textbackslash\textbackslash s stands for \textbf{one } charater whitespace
\item
  The dot ``.'' \textbf{matches any character}
\item
  So, to match a literal dot ``.'' in regex, we need two backslashes then dot \textbackslash\textbackslash.
\item
  The star ``*'' stands for \textbf{0 or more} instances of the previous character
\item
  The plus sign ``+'' stands for \textbf{1 or more} instances of the previous character
\item
  The question mark ``?'' stands for \textbf{0 or one} instance of the previous character
\item
  () () ``\textbackslash\textbackslash1'' capture le groupe de la parenthèse 1 et ``\textbackslash\textbackslash2'' capture le groupe de la parenthèse 2
\end{itemize}

Separate and extract function are from tidyr package.\\
In \textbf{extract}, you can use regex to split a string.

\begin{Shaded}
\begin{Highlighting}[]
\FunctionTok{library}\NormalTok{(dplyr)}
\FunctionTok{library}\NormalTok{(tidyr)}
\NormalTok{s }\OtherTok{\textless{}{-}} \FunctionTok{c}\NormalTok{(}\StringTok{"5\textquotesingle{}6"}\NormalTok{, }\StringTok{"6\textquotesingle{}4"}\NormalTok{)}
\NormalTok{tab }\OtherTok{\textless{}{-}} \FunctionTok{data.frame}\NormalTok{(}\AttributeTok{x =}\NormalTok{ s)}

\NormalTok{tab }\SpecialCharTok{\%\textgreater{}\%} \FunctionTok{separate}\NormalTok{(x,}\FunctionTok{c}\NormalTok{(}\StringTok{"feet"}\NormalTok{,}\StringTok{"inches"}\NormalTok{),}\AttributeTok{sep=}\StringTok{"\textquotesingle{}"}\NormalTok{)}
\end{Highlighting}
\end{Shaded}

\begin{verbatim}
##   feet inches
## 1    5      6
## 2    6      4
\end{verbatim}

\begin{Shaded}
\begin{Highlighting}[]
\NormalTok{tab }\SpecialCharTok{\%\textgreater{}\%} \FunctionTok{extract}\NormalTok{(x,}\FunctionTok{c}\NormalTok{(}\StringTok{"feet"}\NormalTok{,}\StringTok{"inches"}\NormalTok{), }\AttributeTok{regex =} \StringTok{"(}\SpecialCharTok{\textbackslash{}\textbackslash{}}\StringTok{d)\textquotesingle{}(}\SpecialCharTok{\textbackslash{}\textbackslash{}}\StringTok{d\{1,2\})"}\NormalTok{)}
\end{Highlighting}
\end{Shaded}

\begin{verbatim}
##   feet inches
## 1    5      6
## 2    6      4
\end{verbatim}

\hypertarget{creating-samples}{%
\section{Creating samples}\label{creating-samples}}

set.seed(1)
sample()

\hypertarget{about-dates}{%
\section{About dates}\label{about-dates}}

\begin{itemize}
\item
  Sys.time() from base R returns current date/time.
\item
  Extract year of date (with the field already of type date)\\
  with function year from lubridate
  library(lubridate)\\
  df \textless- df \%\textgreater\% mutate(annee=year(mois))
\end{itemize}

\hypertarget{useful-packages-or-datasets}{%
\section{Useful packages or datasets}\label{useful-packages-or-datasets}}

\begin{itemize}
\item
  gapminder
  library(gapminder)\\
  data(``gapminder'')
\item
  tidyr :
  pivot\_wider and pivot\_longer are from tidyverse (tidyr)
\end{itemize}

\hypertarget{useful-libraries}{%
\section{Useful libraries}\label{useful-libraries}}

\begin{itemize}
\item
  recomandation of HWickham:
  In shiny, If you want greater control over the output of dataTableOutput(), I highly recommend the reactable package by Greg Lin.
\item
  It maybe possible to extract a table from a pdf with pdftools\\
  Not tested myself
  library(``pdftools'')
  temp\_file \textless- tempfile()
  url \textless- ``\url{https://www.pnas.org/action/downloadSupplement?doi=10.1073\%2Fpnas.1510159112\&file=pnas.201510159SI.pdf}''
  download.file(url, temp\_file)
  txt \textless- pdf\_text(temp\_file)
  file.remove(temp\_file)
\end{itemize}

\hypertarget{learning-python}{%
\chapter{Learning Python}\label{learning-python}}

\begin{itemize}
\tightlist
\item
  shebang line
  \#!/usr/bin/env python3
\end{itemize}

\hypertarget{learning-gitgithub}{%
\chapter{Learning Git/Github}\label{learning-gitgithub}}

\begin{itemize}
\item
  git config user.name ``my\_name''
\item
  git config user.email ``\href{mailto:me@example.com}{\nolinkurl{me@example.com}}''
\item
  git config -- global user.name ``my\_name''\\
  -\textgreater{} set the value of the username for all git repos\\
  whereas if ``git config'' without global you set it up for the current directory
\item
  git init -\textgreater{} when in the directory which you want to set under git control (initialize a new repo)
\item
  git add myfile -\textgreater{} stagge myfile (place it in the stagging area)
\item
  git commit -m ``my message for this commit''
\item
  git config -l
\item
  git status -\textgreater{} check current state
\item
  three status for tracked files : modified/stagged/commited
\item
  in order to vizualize all \textbf{the commits} (not all the modifications) which were made :\\
  git log
\item
  git add -p
  a way to review changes before adding them
  git will show us which files were not stagged and ask us if we want to commit
\item
  git log -p gives more informations in a viewer\\
  the -p comes from patch
  you can see differences line by line
  you can quit the viewer typing q as with less viewer
\item
  git log --stat
  extra info (how many lines you have added or remove)
\item
  git show `commit\_id'\\
  git show takes a commit id as a parameter
\item
  git -stats
\item
  Admit you modified a file readme.txt which is under version control.\\
  You can see the modifications since the previous version with this command line:\\
  git diff readme.txt
\item
  Add a file to .gitignore in order it is not tracked anymore(?)
  echo 01-Learning-R.Rmd \textgreater{} .gitignore
  echo .RData \textgreater\textgreater{} .gitignore
  after modifying .gitignore you need to stagge (git add) and commit (git commit) it.
\item
  git commit -a -m `message for the commit'
  when you want to commit only the modifications \textbf{super useful}
  (a is for only modified files / m is for message)
\item
  git rm filename\\
  after this you must commit in order the changes to be taken into account
\item
  git mv filename in order to move or rename a file\\
  git mv old\_name new\_name
  after this need to commit
\item
  git diff -u
\item
  git diff only shows unstagged changes by default
\item
  git diff --stagged to see the changes that are staged but not commited
\item
  git
\item
  git checkout ``commit\_id''
  roll back to a previous version
\item
  git reset
  remove from staging area
\item
  git commit --amend
  to modify a commit
\end{itemize}

\hypertarget{footnotes-and-citations}{%
\chapter{Footnotes and citations}\label{footnotes-and-citations}}

\hypertarget{footnotes}{%
\section{Footnotes}\label{footnotes}}

Footnotes are put inside the square brackets after a caret \texttt{\^{}{[}{]}}. Like this one \footnote{This is a footnote.}.

\hypertarget{citations}{%
\section{Citations}\label{citations}}

Reference items in your bibliography file(s) using \texttt{@key}.

For example, we are using the \textbf{bookdown} package \citep{R-bookdown} (check out the last code chunk in index.Rmd to see how this citation key was added) in this sample book, which was built on top of R Markdown and \textbf{knitr} \citep{xie2015} (this citation was added manually in an external file book.bib).
Note that the \texttt{.bib} files need to be listed in the index.Rmd with the YAML \texttt{bibliography} key.

The RStudio Visual Markdown Editor can also make it easier to insert citations: \url{https://rstudio.github.io/visual-markdown-editing/\#/citations}

\hypertarget{learning-markdown}{%
\chapter{Learning Markdown}\label{learning-markdown}}

In markdown you need to escape twice the backslash in order to display two backslashes\\
So what you see here, I wrote it with \textbf{four} not just three : \textbf{\textbackslash\textbackslash{}}

To write a list, you must write a star * followed by a coma a the beginning of a line.
Before the list starts you need a blankline and same at the end of the list otherwise Mardown won't recognize it.

To introduce a return to the ligne, you need not only to type return in Markdown, but also to make the line followed by two spaces.

To make a few words bold you need to surrender it with two stars both sides.\\
It is **bold** gives : It is \textbf{bold}

\hypertarget{learning-linux-commands}{%
\chapter{Learning linux commands}\label{learning-linux-commands}}

\begin{itemize}
\item
  git --version
\item
  mkdir -\textgreater{} create a directory
\item
  cat to read a file
\item
  or less (type q in order to exit less viewer)
  why less? because previous version of less was more :)
\item
  write in a file :\\
  echo toto et tata \textgreater{} toto.txt
\end{itemize}

echo toto et titi \textgreater{} titi.txt

\begin{itemize}
\item
  differences between two files:\\
  diff toto.txt titi.txt\\
  or diff -u toto.txt titi.txt
\item
  diff -u is more readable than simple diff command.
\item
  Create a diff file:\\
  diff -u toto.txt titi.txt \textgreater{} change.diff
\item
  Patch the .diff file:\\
  patch titi.txt \textless{} change.diff
\item
  Clear the console:\\
  just as in Rstudio ctrl+l
  or typing ``clear'' and then enter in the console.
  both works
\item
  Content of a directory:\\
  dir or ls : both works.
\item
  Content of a directory including hidden files:\\
  dir -a
  ls -a
  from the help of ls : '' -a, --all do not ignore entries starting with.''
\item
  Add the options l to see rights on the files:
  ls -la
\item
  Get the help in git bash on windows:\\
  function --help
  example: ls --help
\item
  Make a file executable:\\
  chmod +x filename
\item
  Open a file with nano :\\
  nano my\_file.txt
\item
  Save changes made to a file in nano:\\
  ctrl+o + Enter + ctrl+x
\item
  `cd -' in order to come back to previous directory
\item
  set a file in executable mode
  chmod +x file\_name
\end{itemize}

\hypertarget{statistics-with-r}{%
\chapter{Statistics with R}\label{statistics-with-r}}

\begin{Shaded}
\begin{Highlighting}[]
\NormalTok{beads }\OtherTok{\textless{}{-}} \FunctionTok{rep}\NormalTok{(}\FunctionTok{c}\NormalTok{(}\StringTok{"red"}\NormalTok{,}\StringTok{"blue"}\NormalTok{), }\AttributeTok{times =} \FunctionTok{c}\NormalTok{(}\DecValTok{2}\NormalTok{,}\DecValTok{3}\NormalTok{))}
\NormalTok{beads}
\end{Highlighting}
\end{Shaded}

\begin{verbatim}
## [1] "red"  "red"  "blue" "blue" "blue"
\end{verbatim}

\begin{Shaded}
\begin{Highlighting}[]
\CommentTok{\# pick a bead at random}
\FunctionTok{sample}\NormalTok{(beads,}\DecValTok{1}\NormalTok{)}
\end{Highlighting}
\end{Shaded}

\begin{verbatim}
## [1] "blue"
\end{verbatim}

\begin{Shaded}
\begin{Highlighting}[]
\CommentTok{\# evaluate the probability of drawing a blue at random}
\FunctionTok{mean}\NormalTok{(beads }\SpecialCharTok{==} \StringTok{"blue"}\NormalTok{)}
\end{Highlighting}
\end{Shaded}

\begin{verbatim}
## [1] 0.6
\end{verbatim}

\begin{Shaded}
\begin{Highlighting}[]
\NormalTok{B }\OtherTok{\textless{}{-}} \DecValTok{10000}

\CommentTok{\# Nota : if you want the "same random" each time }
\CommentTok{\# you draw the sample, you need to set a seed}
\FunctionTok{set.seed}\NormalTok{(}\DecValTok{1}\NormalTok{)}
\CommentTok{\# for example}



\NormalTok{events }\OtherTok{\textless{}{-}} \FunctionTok{replicate}\NormalTok{(B,}\FunctionTok{sample}\NormalTok{(beads,}\DecValTok{1}\NormalTok{))}
\NormalTok{tab }\OtherTok{\textless{}{-}} \FunctionTok{table}\NormalTok{(events)}
\CommentTok{\# calculate probability of each events}
\FunctionTok{prop.table}\NormalTok{(tab)}
\end{Highlighting}
\end{Shaded}

\begin{verbatim}
## events
##   blue    red 
## 0.6028 0.3972
\end{verbatim}

\begin{Shaded}
\begin{Highlighting}[]
\CommentTok{\# note that with this monte carlo simulation we have }
\CommentTok{\# quite the same probability for blue events as with}
\CommentTok{\# mean(beads =="blue")}
\end{Highlighting}
\end{Shaded}

by default, sample samples without replacement.
it means, it you try

\begin{Shaded}
\begin{Highlighting}[]
\FunctionTok{sample}\NormalTok{(beads,}\DecValTok{5}\NormalTok{)}
\end{Highlighting}
\end{Shaded}

\begin{verbatim}
## [1] "blue" "red"  "red"  "blue" "blue"
\end{verbatim}

\begin{Shaded}
\begin{Highlighting}[]
\CommentTok{\# it works, but if you try}
\CommentTok{\# sample(beads,6)}
\CommentTok{\# it fails because there are only 6 beads}
\end{Highlighting}
\end{Shaded}

so sample by default is equivalent to
sample(beads,5, replace = FALSE)
but we can use sample \textbf{with} replacement.

\begin{Shaded}
\begin{Highlighting}[]
\CommentTok{\# so this is working:}
\FunctionTok{sample}\NormalTok{(beads,}\DecValTok{6}\NormalTok{,}\AttributeTok{replace =} \ConstantTok{TRUE}\NormalTok{)}
\end{Highlighting}
\end{Shaded}

\begin{verbatim}
## [1] "blue" "red"  "red"  "red"  "blue" "red"
\end{verbatim}

\begin{Shaded}
\begin{Highlighting}[]
\CommentTok{\# and we can use simply sample with replacement}
\CommentTok{\# juste like with replicate}

\NormalTok{events2 }\OtherTok{\textless{}{-}} \FunctionTok{sample}\NormalTok{(beads,B, }\AttributeTok{replace =} \ConstantTok{TRUE}\NormalTok{)}
\NormalTok{tab2 }\OtherTok{\textless{}{-}} \FunctionTok{table}\NormalTok{(events2)}
\FunctionTok{prop.table}\NormalTok{(tab2)}
\end{Highlighting}
\end{Shaded}

\begin{verbatim}
## events2
##   blue    red 
## 0.6026 0.3974
\end{verbatim}

\begin{Shaded}
\begin{Highlighting}[]
\CommentTok{\# the result is almost the same}
\end{Highlighting}
\end{Shaded}

discrete and continuous variables
example:
discrete : flip of a coin
outcome from the roll of a die

web site traffic on a given day \textgreater{} particular discrete variable (no upper bound \textgreater{} poisson)
same for the number of people clicking on an add

continuous :
BMI (body mass index) of a subject four years after a baseline measurement
hypertension status but could be modelled as discrete (1 hypertension/0 no hypertension)

\begin{itemize}
\item
  probability mass function (pmf)
  for discrete variable \textgreater{} assign a probability for each value they can take
  1-must always be larger or equal to 0
  (prob is number bet 0 abd 1)
  2- the sum of the poss values has to add up to 1
\item
  examples of pmf :
  binomial, canonical (flipping a cooin)
\end{itemize}

Bernouli = flip a coin
X = 0 tails
X = 1 represents heads

Two broad flavors of inference :
* frequency, which uses ``long run proportion of times an event occurs in \textbf{independent}, identically distributed repetitions''
* he second is Bayesian in which the probability estimate for a hypothesis is updated as additional evidence is acquired

\begin{itemize}
\item
  If A and B are two \textbf{independent} events then the probability of them both occurring is the product of the probabilities. P(A\&B) = P(A) * P(B)
\item
  Suppose you rolled the fair die twice. What is the probability of rolling the same number two times in a row? Since we don't care what the outcome of the first roll is, its probability is 1. The second roll of the dice has to match the outcome of the first, so that has a probability of 1/6. The probability of both events occurring is 1 * 1/6.
\item
  The probability of at least one of two events, A and B, occurring is the sum of their individual
  probabilities minus the probability of their intersection. P(A U B) = P(A) + P(B) - P(A\&B).
\end{itemize}

\hypertarget{learning-shiny}{%
\chapter{Learning Shiny}\label{learning-shiny}}

library(shiny)

\begin{itemize}
\item
  run shinyapp shortcut:\\
  ctrl + shift + enter
  not ctrl + shift + K (this is to knit Rmarkdown's documents)
\item
  To have a look on which html tags are available in Shiny:
  ?builder
\item
  Ctrl + U to see the html content of a web page \textgreater{} it opens the html code in a new page
\item
  Alt + s + w + s
  or via the menu Session \textgreater{} setwdir \textgreater{} To source file loc
\end{itemize}

runApp()

shinyUI()

shinyServer()

\hypertarget{outputs}{%
\section{outputs}\label{outputs}}

\hypertarget{text}{%
\subsection{Text}\label{text}}

\begin{itemize}
\item
  renderText() combines the result into a single string, and is usually paired with textOutput()
\item
  renderPrint() prints the result, as if you were in an R console, and is usually paired with verbatimTextOutput()
\end{itemize}

\hypertarget{tables}{%
\subsection{Tables}\label{tables}}

tableOutput() and renderTable() render a static table of data, showing all the data at once.

dataTableOutput() and renderDataTable() render a dynamic table, showing a fixed number of rows along with controls to change which rows are visible.

\hypertarget{plots}{%
\subsection{Plots}\label{plots}}

You can display any type of R graphic (base, ggplot2, or otherwise) with plotOutput() and renderPlot()

\hypertarget{basic-reactivity}{%
\section{Basic reactivity}\label{basic-reactivity}}

\hypertarget{server-function}{%
\subsection{Server function}\label{server-function}}

\hypertarget{input}{%
\subsubsection{Input}\label{input}}

It must be a simple string that contains only letters, numbers, and underscores (no spaces, dashes, periods, or other special characters allowed!). Name it like you would name a variable in R.

It must be unique. If it's not unique, you'll have no way to refer to this control in your server function!

Most input functions have a second parameter called label. This is used to create a human-readable label for the control. Shiny doesn't place any restrictions on this string, but you'll need to carefully think about it to make sure that your app is usable by humans! The third parameter is typically value, which, where possible, lets you set the default value. The remaining parameters are unique to the control

sliderInput(): If you supply a length-2 numeric vector for the default value of sliderInput(), you get a ``range'' slider with two ends.
exemple : sliderInput(``rng'', ``Range'', value = c(10, 20), min = 0, max = 100)

Unlike a typical list, input objects are read-only. If you attempt to modify an input inside the server function, you'll get an error.

To read from an input, you must be in a reactive context created by a function like renderText() or reactive()

\hypertarget{output}{%
\subsubsection{Output}\label{output}}

You always use the output object in concert with a render function, as in the following simple example.

Link toward vers Mastering shiny from Hadley Wickham

Each \textbf{render\{Type\} }function is designed to produce a particular type of output (e.g.~text, tables, and plots), and is \emph{often paired} with a \textbf{\{type\}Output} function. For example, in this app, renderPrint() is paired with verbatimTextOutput() to display a statistical summary with fixed-width (verbatim) text, and renderTable() is paired with tableOutput() to show the input data in a table.

server ui
textInput
selectInput
sliderInput
passwordInput
textAreaInput
numericInput
dateInput
dateRangeInput
radioButtons
checkboxGroupInput
fileInput
actionButton

renderPrint() verbatimTextOutput()
renderTable() tableOutput()
renderDataTable() dataTableOutput()
renderPlot(,res = 96) plotOutput()\\
renderText() textOutput()

output\(name xxxOutput("name") input\)id\_selected selectInput(``id\_selected'', label)

exemple of radioButton
animals \textless- c(``dog'', ``cat'', ``mouse'', ``bird'', ``other'', ``I hate animals'')
radioButtons(``animal'', ``What's your favourite animal?'', animals)

Allow the user to select multiple elements
selectInput(
``state'', ``What's your favourite state?'', state.name,
multiple = TRUE)

There's no way to select multiple values with radio buttons, but there's an alternative that's conceptually similar: checkboxGroupInput().

ui \textless- fluidPage(
checkboxGroupInput(``animal'', ``What animals do you like?'', animals)
)

You create a reactive expression by wrapping a block of code in reactive(\{\ldots\}) and assigning it to a variable, and you use a reactive expression by calling it \textbf{like a function} that is the name of the reactive expression we created followed by parentheses.
Note that the \{\} are only required in render functions if need to run multiple lines of code. As you'll learn shortly, you should do as little computation in your render functions as possible, which means you can often omit them

To create a reactive expression, we call reactive() and assign the results to a variable. To later use the expression, we call the variable like it's a function.

exemple:\\
server \textless- function(input, output, session) \{
x1 \textless- reactive(rnorm(input\(n1, input\)mean1, input\$sd1))

Most simple exemple of reactive expression:\\
server \textless- function(input, output, session) \{
string \textless- reactive(paste0(``Hello'', input\(name, "!"))  output\)greeting \textless- renderText(string())
\}

Just like any other R function, when the server function is called it creates a new local environment that is independent of every other invocation of the function. This allows each session to have a unique state, as well as isolating the variables created inside the function.

the order in which reactive code is run is determined only by the reactive graph, not by its layout in the server function.

\hypertarget{learning-html}{%
\chapter{Learning html}\label{learning-html}}

Install visual studio and live server?

\begin{Shaded}
\begin{Highlighting}[]
\CommentTok{\#\textless{}DOCTYPE! html\textgreater{}}
\CommentTok{\#\textless{}html lang="en"\textgreater{}}


\CommentTok{\#\textless{}\textbackslash{}html\textgreater{}}
\end{Highlighting}
\end{Shaded}

ctr + u to see the content of a web page

form tag includes input tag(s)
form as actions such as action or method
input is a self closing tag
input can take many attributes such as value placeholder, readonly required name max min

pour valider le code :
\url{https://validator.w3.org/}

pour valider l'accessibilité :
swave mais uniquement avec une url publiée\ldots{}

tag is for metadata

section of the page that links to other pages or to parts within the page

exemples

\begin{verbatim}
    <input  type ="text" placeholder="ceci est mon exemple"/>

</form>
\end{verbatim}

\begin{verbatim}
    <input  type ="text" value="valeur_ini"/>
        <input type="date":>

    </form>
</main>
<footer>

</footer>
\end{verbatim}

\begin{itemize}
\tightlist
\item
  tag img
\end{itemize}

\&copy to display copyright
\&gt greater than
\&lt less than
\&nbsp to display blank space
\&amp display le ésperluette

\begin{itemize}
\item
  tag is for informations at the bottom of the page:
  warning footer doesn't place the information at the bottom
  copyrights
  links to social medias
  related documents
\item
  more semantics than
\end{itemize}

Second page
or Second page
or link to a different location in the page (link with an ID)
History section

leads to

\hypertarget{history}{}

the image is the link

blabla to open the link into a new tab

.ext et on vient coller \#t= 5,25 \textgreater{} joue la vidéo ou l'audio de la seconde 5 à la seconde 25

\begin{itemize}
\tightlist
\item
  css
\end{itemize}

h1 \{color:\}
selector property colon value of the property

\begin{itemize}
\item
  element selectors
  apply style to all similar elements
  eg to

  paragraphs
\item
  class selectors
  apply style to all elements with a specific class
\item
  id selectors
  apply style to single elements through the id association
\item
  attribute selectors
  applys styles to elements with a specific attribute
  such as all the anchors tags wich have the target attribute
\end{itemize}

exemples:

\begin{itemize}
\tightlist
\item
  simple selectors
\item
  tag selectors
\end{itemize}

\begin{itemize}
\tightlist
\item
  class selectors (class = a group of elements)
\end{itemize}

.redBackground \{
background-color : red
\}

\begin{itemize}
\tightlist
\item
  id selectors
  match a single element with a special ID attribute
  (each and every html element has a unique id)
\end{itemize}

This is third h1

\begin{itemize}
\tightlist
\item
  universal selector
  will math all elements in the page
\end{itemize}

\begin{itemize}
\item
  CSS combined selectors also called combinators
\item
  descendant selector
\end{itemize}

\begin{itemize}
\tightlist
\item
  child selector
  apply only to \textbf{direct children} not to all descendants

  parent \textgreater{} child \{
  color:blue;
  \}
\end{itemize}

div \textgreater{} h1 \{
color:red;
\}
s'applique seulement aux h1 directement sous un div. S'il y a un tag intermédiaire (notion de dépendant), ça ne s'appliquera pas.

\begin{itemize}
\tightlist
\item
  adjacent sibling selector :

  h1 + h2 \{
  color:blue;
  \}
\end{itemize}

\begin{itemize}
\item
  general sibling selector :
  h1 \textasciitilde{} h2 \{
  color:blue;
  \}
\item
  pseudo class selectors :
  target specific elements based on their current states such as hovered, click\ldots{}
\item
  :hover pseudo class used to style an element when the user hovers over it with a mouse
  example
  Click me
\end{itemize}

\begin{itemize}
\tightlist
\item
  :activate
  when a button is clicked
\end{itemize}

Click me

\begin{itemize}
\tightlist
\item
  :visited
  when a link had been visited
\end{itemize}

Click here

\begin{itemize}
\tightlist
\item
  :focus
\end{itemize}

\begin{itemize}
\item
  :nth-child
  to select elements based on their position in the parent element

  Apply the style only on the first paragraph
\item
  :nth-child(2n) : apply only to even elements
\item
  :nth-child(2n+1) : apply only to odd elements
\item
  :nth-child(odd) works to
\item
  :nth-child(even)
\item
  :first-child
  to select the first child element of a parent
\item
  :last-child
  to select the last child element of a parent
\end{itemize}

reprendre à pseudo elements après pseudo class selectors part 2

\begin{itemize}
\item
  pseudo elements
  to style specified parts of an element
  you must include it inside your tag
  eg first letter of a paragraph
  denoted by two semi-colons
\item
  ::before
\end{itemize}

::before

\begin{itemize}
\item
  ::after
\item
  ::first-ligne
\item
  attribute selectors
  style elements based on the presence or the value of their attributes
  denoted by square brackets
  in combination with other selectors
\item
  example:

  Internal1
  Google
\item
  example:
\end{itemize}

\hypertarget{sql-tips}{%
\chapter{SQL tips}\label{sql-tips}}

\begin{itemize}
\tightlist
\item
  modify a database entry according to criterion
\end{itemize}

UPDATE ``anapaie\_rub\_6411\_silae'' SET ``compte'' = `64110000'
WHERE ``code\_etb'' = `6001' AND ``code\_rub'' IN (`I03',`I04')

\hypertarget{keyboard-shortcuts}{%
\chapter{Keyboard shortcuts}\label{keyboard-shortcuts}}

\begin{itemize}
\tightlist
\item
  F12 = save as in MS Office
\end{itemize}

  \bibliography{book.bib,packages.bib}

\end{document}
