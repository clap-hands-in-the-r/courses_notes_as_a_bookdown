% Options for packages loaded elsewhere
\PassOptionsToPackage{unicode}{hyperref}
\PassOptionsToPackage{hyphens}{url}
%
\documentclass[
]{book}
\usepackage{amsmath,amssymb}
\usepackage{lmodern}
\usepackage{iftex}
\ifPDFTeX
  \usepackage[T1]{fontenc}
  \usepackage[utf8]{inputenc}
  \usepackage{textcomp} % provide euro and other symbols
\else % if luatex or xetex
  \usepackage{unicode-math}
  \defaultfontfeatures{Scale=MatchLowercase}
  \defaultfontfeatures[\rmfamily]{Ligatures=TeX,Scale=1}
\fi
% Use upquote if available, for straight quotes in verbatim environments
\IfFileExists{upquote.sty}{\usepackage{upquote}}{}
\IfFileExists{microtype.sty}{% use microtype if available
  \usepackage[]{microtype}
  \UseMicrotypeSet[protrusion]{basicmath} % disable protrusion for tt fonts
}{}
\makeatletter
\@ifundefined{KOMAClassName}{% if non-KOMA class
  \IfFileExists{parskip.sty}{%
    \usepackage{parskip}
  }{% else
    \setlength{\parindent}{0pt}
    \setlength{\parskip}{6pt plus 2pt minus 1pt}}
}{% if KOMA class
  \KOMAoptions{parskip=half}}
\makeatother
\usepackage{xcolor}
\usepackage{color}
\usepackage{fancyvrb}
\newcommand{\VerbBar}{|}
\newcommand{\VERB}{\Verb[commandchars=\\\{\}]}
\DefineVerbatimEnvironment{Highlighting}{Verbatim}{commandchars=\\\{\}}
% Add ',fontsize=\small' for more characters per line
\usepackage{framed}
\definecolor{shadecolor}{RGB}{248,248,248}
\newenvironment{Shaded}{\begin{snugshade}}{\end{snugshade}}
\newcommand{\AlertTok}[1]{\textcolor[rgb]{0.94,0.16,0.16}{#1}}
\newcommand{\AnnotationTok}[1]{\textcolor[rgb]{0.56,0.35,0.01}{\textbf{\textit{#1}}}}
\newcommand{\AttributeTok}[1]{\textcolor[rgb]{0.77,0.63,0.00}{#1}}
\newcommand{\BaseNTok}[1]{\textcolor[rgb]{0.00,0.00,0.81}{#1}}
\newcommand{\BuiltInTok}[1]{#1}
\newcommand{\CharTok}[1]{\textcolor[rgb]{0.31,0.60,0.02}{#1}}
\newcommand{\CommentTok}[1]{\textcolor[rgb]{0.56,0.35,0.01}{\textit{#1}}}
\newcommand{\CommentVarTok}[1]{\textcolor[rgb]{0.56,0.35,0.01}{\textbf{\textit{#1}}}}
\newcommand{\ConstantTok}[1]{\textcolor[rgb]{0.00,0.00,0.00}{#1}}
\newcommand{\ControlFlowTok}[1]{\textcolor[rgb]{0.13,0.29,0.53}{\textbf{#1}}}
\newcommand{\DataTypeTok}[1]{\textcolor[rgb]{0.13,0.29,0.53}{#1}}
\newcommand{\DecValTok}[1]{\textcolor[rgb]{0.00,0.00,0.81}{#1}}
\newcommand{\DocumentationTok}[1]{\textcolor[rgb]{0.56,0.35,0.01}{\textbf{\textit{#1}}}}
\newcommand{\ErrorTok}[1]{\textcolor[rgb]{0.64,0.00,0.00}{\textbf{#1}}}
\newcommand{\ExtensionTok}[1]{#1}
\newcommand{\FloatTok}[1]{\textcolor[rgb]{0.00,0.00,0.81}{#1}}
\newcommand{\FunctionTok}[1]{\textcolor[rgb]{0.00,0.00,0.00}{#1}}
\newcommand{\ImportTok}[1]{#1}
\newcommand{\InformationTok}[1]{\textcolor[rgb]{0.56,0.35,0.01}{\textbf{\textit{#1}}}}
\newcommand{\KeywordTok}[1]{\textcolor[rgb]{0.13,0.29,0.53}{\textbf{#1}}}
\newcommand{\NormalTok}[1]{#1}
\newcommand{\OperatorTok}[1]{\textcolor[rgb]{0.81,0.36,0.00}{\textbf{#1}}}
\newcommand{\OtherTok}[1]{\textcolor[rgb]{0.56,0.35,0.01}{#1}}
\newcommand{\PreprocessorTok}[1]{\textcolor[rgb]{0.56,0.35,0.01}{\textit{#1}}}
\newcommand{\RegionMarkerTok}[1]{#1}
\newcommand{\SpecialCharTok}[1]{\textcolor[rgb]{0.00,0.00,0.00}{#1}}
\newcommand{\SpecialStringTok}[1]{\textcolor[rgb]{0.31,0.60,0.02}{#1}}
\newcommand{\StringTok}[1]{\textcolor[rgb]{0.31,0.60,0.02}{#1}}
\newcommand{\VariableTok}[1]{\textcolor[rgb]{0.00,0.00,0.00}{#1}}
\newcommand{\VerbatimStringTok}[1]{\textcolor[rgb]{0.31,0.60,0.02}{#1}}
\newcommand{\WarningTok}[1]{\textcolor[rgb]{0.56,0.35,0.01}{\textbf{\textit{#1}}}}
\usepackage{longtable,booktabs,array}
\usepackage{calc} % for calculating minipage widths
% Correct order of tables after \paragraph or \subparagraph
\usepackage{etoolbox}
\makeatletter
\patchcmd\longtable{\par}{\if@noskipsec\mbox{}\fi\par}{}{}
\makeatother
% Allow footnotes in longtable head/foot
\IfFileExists{footnotehyper.sty}{\usepackage{footnotehyper}}{\usepackage{footnote}}
\makesavenoteenv{longtable}
\usepackage{graphicx}
\makeatletter
\def\maxwidth{\ifdim\Gin@nat@width>\linewidth\linewidth\else\Gin@nat@width\fi}
\def\maxheight{\ifdim\Gin@nat@height>\textheight\textheight\else\Gin@nat@height\fi}
\makeatother
% Scale images if necessary, so that they will not overflow the page
% margins by default, and it is still possible to overwrite the defaults
% using explicit options in \includegraphics[width, height, ...]{}
\setkeys{Gin}{width=\maxwidth,height=\maxheight,keepaspectratio}
% Set default figure placement to htbp
\makeatletter
\def\fps@figure{htbp}
\makeatother
\setlength{\emergencystretch}{3em} % prevent overfull lines
\providecommand{\tightlist}{%
  \setlength{\itemsep}{0pt}\setlength{\parskip}{0pt}}
\setcounter{secnumdepth}{5}
\usepackage{booktabs}
\ifLuaTeX
  \usepackage{selnolig}  % disable illegal ligatures
\fi
\usepackage[]{natbib}
\bibliographystyle{plainnat}
\IfFileExists{bookmark.sty}{\usepackage{bookmark}}{\usepackage{hyperref}}
\IfFileExists{xurl.sty}{\usepackage{xurl}}{} % add URL line breaks if available
\urlstyle{same} % disable monospaced font for URLs
\hypersetup{
  pdftitle={A Minimal Book Example},
  pdfauthor={John Doe},
  hidelinks,
  pdfcreator={LaTeX via pandoc}}

\title{A Minimal Book Example}
\author{John Doe}
\date{2023-08-30}

\begin{document}
\maketitle

{
\setcounter{tocdepth}{1}
\tableofcontents
}
\hypertarget{about}{%
\chapter{About}\label{about}}

This is a \emph{sample} book written in \textbf{Markdown}. You can use anything that Pandoc's Markdown supports; for example, a math equation \(a^2 + b^2 = c^2\).

\hypertarget{usage}{%
\section{Usage}\label{usage}}

Each \textbf{bookdown} chapter is an .Rmd file, and each .Rmd file can contain one (and only one) chapter. A chapter \emph{must} start with a first-level heading: \texttt{\#\ A\ good\ chapter}, and can contain one (and only one) first-level heading.

Use second-level and higher headings within chapters like: \texttt{\#\#\ A\ short\ section} or \texttt{\#\#\#\ An\ even\ shorter\ section}.

The \texttt{index.Rmd} file is required, and is also your first book chapter. It will be the homepage when you render the book.

\hypertarget{render-book}{%
\section{Render book}\label{render-book}}

You can render the HTML version of this example book without changing anything:

\begin{enumerate}
\def\labelenumi{\arabic{enumi}.}
\item
  Find the \textbf{Build} pane in the RStudio IDE, and
\item
  Click on \textbf{Build Book}, then select your output format, or select ``All formats'' if you'd like to use multiple formats from the same book source files.
\end{enumerate}

Or build the book from the R console:

\begin{Shaded}
\begin{Highlighting}[]
\NormalTok{bookdown}\SpecialCharTok{::}\FunctionTok{render\_book}\NormalTok{()}
\end{Highlighting}
\end{Shaded}

To render this example to PDF as a \texttt{bookdown::pdf\_book}, you'll need to install XeLaTeX. You are recommended to install TinyTeX (which includes XeLaTeX): \url{https://yihui.org/tinytex/}.

\hypertarget{preview-book}{%
\section{Preview book}\label{preview-book}}

As you work, you may start a local server to live preview this HTML book. This preview will update as you edit the book when you save individual .Rmd files. You can start the server in a work session by using the RStudio add-in ``Preview book'', or from the R console:

\begin{Shaded}
\begin{Highlighting}[]
\NormalTok{bookdown}\SpecialCharTok{::}\FunctionTok{serve\_book}\NormalTok{()}
\end{Highlighting}
\end{Shaded}

\hypertarget{learning-r}{%
\chapter{Learning R}\label{learning-r}}

This is notes about learning R.

Please build this simple boook!!!!! ahhhhhhhhhhhhha

\hypertarget{about-stringr-package}{%
\section{About stringr package}\label{about-stringr-package}}

\textbf{Stringr} is more coherent than base R functions for strings treatments.\\
Stringr functions always begin with prefix \textbf{str\_} ; the first argument is always the string you want to treat. And then comes the pattern you want to identify.

Most common and useful functions in Stringr :

\begin{itemize}
\tightlist
\item
  str\_detect() -\textgreater{} returns a logical vector (a vector of TRUE and FALSE)
\item
  str\_subset()
\item
  str\_view()
\item
  str\_view\_all()
\item
  str\_replace()
\item
  str\_replace\_all()
\item
  str\_split()
\item
  str\_trim()
\item
  str\_to\_lower()
\end{itemize}

\hypertarget{about-regex-in-r}{%
\section{About Regex in R}\label{about-regex-in-r}}

\hypertarget{special-characters}{%
\subsection{Special characters}\label{special-characters}}

\begin{itemize}
\tightlist
\item
  \textbackslash\textbackslash d stands for \textbf{one of any digit 0,1,2, up to 9}
\item
  \textbackslash\textbackslash s stands for \textbf{one } charater whitespace
\item
  The dot ``.'' \textbf{matches any character}
\item
  So, to match a literal dot ``.'' in regex, we need two backslashes then dot \textbackslash\textbackslash.
\item
  The star ``*'' stands for \textbf{0 or more} instances of the previous character
\item
  The plus sign ``+'' stands for \textbf{1 or more} instances of the previous character
\item
  The question mark ``?'' stands for \textbf{0 or one} instance of the previous character
\item
  () () ``\textbackslash\textbackslash1'' capture le groupe de la parenthèse 1 et ``\textbackslash\textbackslash2'' capture le groupe de la parenthèse 2
\end{itemize}

Separate and extract function are from tidyr package.\\
In \textbf{extract}, you can use regex to split a string.

\begin{Shaded}
\begin{Highlighting}[]
\FunctionTok{library}\NormalTok{(dplyr)}
\FunctionTok{library}\NormalTok{(tidyr)}
\NormalTok{s }\OtherTok{\textless{}{-}} \FunctionTok{c}\NormalTok{(}\StringTok{"5\textquotesingle{}6"}\NormalTok{, }\StringTok{"6\textquotesingle{}4"}\NormalTok{)}
\NormalTok{tab }\OtherTok{\textless{}{-}} \FunctionTok{data.frame}\NormalTok{(}\AttributeTok{x =}\NormalTok{ s)}

\NormalTok{tab }\SpecialCharTok{\%\textgreater{}\%} \FunctionTok{separate}\NormalTok{(x,}\FunctionTok{c}\NormalTok{(}\StringTok{"feet"}\NormalTok{,}\StringTok{"inches"}\NormalTok{),}\AttributeTok{sep=}\StringTok{"\textquotesingle{}"}\NormalTok{)}
\end{Highlighting}
\end{Shaded}

\begin{verbatim}
##   feet inches
## 1    5      6
## 2    6      4
\end{verbatim}

\begin{Shaded}
\begin{Highlighting}[]
\NormalTok{tab }\SpecialCharTok{\%\textgreater{}\%} \FunctionTok{extract}\NormalTok{(x,}\FunctionTok{c}\NormalTok{(}\StringTok{"feet"}\NormalTok{,}\StringTok{"inches"}\NormalTok{), }\AttributeTok{regex =} \StringTok{"(}\SpecialCharTok{\textbackslash{}\textbackslash{}}\StringTok{d)\textquotesingle{}(}\SpecialCharTok{\textbackslash{}\textbackslash{}}\StringTok{d\{1,2\})"}\NormalTok{)}
\end{Highlighting}
\end{Shaded}

\begin{verbatim}
##   feet inches
## 1    5      6
## 2    6      4
\end{verbatim}

\hypertarget{useful-packages-or-datasets}{%
\section{Useful packages or datasets}\label{useful-packages-or-datasets}}

\begin{itemize}
\tightlist
\item
  gapminder
  library(gapminder)\\
  data(``gapminder'')
\end{itemize}

\hypertarget{useful-libraries}{%
\section{Useful libraries}\label{useful-libraries}}

\begin{itemize}
\tightlist
\item
  It maybe possible to extract a table from a pdf with pdftools\\
  Not tested myself
  library(``pdftools'')
  temp\_file \textless- tempfile()
  url \textless- ``\url{https://www.pnas.org/action/downloadSupplement?doi=10.1073\%2Fpnas.1510159112\&file=pnas.201510159SI.pdf}''
  download.file(url, temp\_file)
  txt \textless- pdf\_text(temp\_file)
  file.remove(temp\_file)
\end{itemize}

\hypertarget{learning-python}{%
\chapter{Learning Python}\label{learning-python}}

\begin{itemize}
\tightlist
\item
  shebang line
  \#!/usr/bin/env python3
\end{itemize}

\hypertarget{learning-gitgithub}{%
\chapter{Learning Git/Github}\label{learning-gitgithub}}

\begin{itemize}
\item
  git config user.name ``my\_name''
\item
  git config user.email ``\href{mailto:me@example.com}{\nolinkurl{me@example.com}}''
\item
  git config -- global user.name ``my\_name''\\
  -\textgreater{} set the value of the username for all git repos\\
  whereas if ``git config'' without global you set it up for the current directory
\item
  git init -\textgreater{} when in the directory which you want to set under git control (initialize a new repo)
\item
  git add myfile -\textgreater{} stagge myfile (place it in the stagging area)
\item
  git commit -m ``my message for this commit''
\item
  git config -l
\item
  git status -\textgreater{} check current state
\item
  three status for tracked files : modified/stagged/commited
\item
  in order to vizualize all \textbf{the commits} (not all the modifications) which were made :\\
  git log
\end{itemize}

Admit you modified a file readme.txt which is under version control.\\
You can see the modifications since the previous version with this command line:\\
git diff readme.txt

\begin{itemize}
\item
  Add a file to .gitignore in order it is not tracked anymore(?)
  echo 01-Learning-R.Rmd \textgreater{} .gitignore
  echo .RData \textgreater\textgreater{} .gitignore
  after modifying .gitignore you need to stagge (git add) and commit (git commit) it.
\item
  git commit -a -m `message for the commit'
  when you want to commit only the modifications
  (a is for only modified files / m is for message)
\item
  git rm filename\\
  after this you must commit
\item
  git mv filename in order to move or rename a file
\end{itemize}

\hypertarget{footnotes-and-citations}{%
\chapter{Footnotes and citations}\label{footnotes-and-citations}}

\hypertarget{footnotes}{%
\section{Footnotes}\label{footnotes}}

Footnotes are put inside the square brackets after a caret \texttt{\^{}{[}{]}}. Like this one \footnote{This is a footnote.}.

\hypertarget{citations}{%
\section{Citations}\label{citations}}

Reference items in your bibliography file(s) using \texttt{@key}.

For example, we are using the \textbf{bookdown} package \citep{R-bookdown} (check out the last code chunk in index.Rmd to see how this citation key was added) in this sample book, which was built on top of R Markdown and \textbf{knitr} \citep{xie2015} (this citation was added manually in an external file book.bib).
Note that the \texttt{.bib} files need to be listed in the index.Rmd with the YAML \texttt{bibliography} key.

The RStudio Visual Markdown Editor can also make it easier to insert citations: \url{https://rstudio.github.io/visual-markdown-editing/\#/citations}

\hypertarget{learning-markdown}{%
\chapter{Learning Markdown}\label{learning-markdown}}

In markdown you need to escape twice the backslash in order to display two backslashes\\
So what you see here, I wrote it with \textbf{four} not just three : \textbf{\textbackslash\textbackslash{}}

To write a list, you must write a star * followed by a coma a the beginning of a line.
Before the list starts you need a blankline and same at the end of the list otherwise Mardown won't recognize it.

To introduce a return to the ligne, you need not only to type return in Markdown, but also to make the line followed by two spaces.

To make a few words bold you need to surrender it with two stars both sides.\\
It is **bold** gives : It is \textbf{bold}

\hypertarget{learning-linux-commands}{%
\chapter{Learning linux commands}\label{learning-linux-commands}}

\begin{itemize}
\item
  git --version
\item
  mkdir -\textgreater{} create a directory
\item
  cat to read a file
\item
  or less (type q in order to exit less viewer)
  why less? because previous version of less was more :)
\item
  write in a file :\\
  echo toto et tata \textgreater{} toto.txt
\end{itemize}

echo toto et titi \textgreater{} titi.txt

\begin{itemize}
\item
  differences between two files:\\
  diff toto.txt titi.txt\\
  or diff -u toto.txt titi.txt
\item
  diff -u is more readable than simple diff command.
\item
  Create a diff file:\\
  diff -u toto.txt titi.txt \textgreater{} change.diff
\item
  Patch the .diff file:\\
  patch titi.txt \textless{} change.diff
\item
  Clear the console:\\
  just as in Rstudio ctrl+l
  or typing ``clear'' and then enter in the console.
  both works
\item
  Content of a directory:\\
  dir or ls : both works.
\item
  Content of a directory including hidden files:\\
  dir -a
  ls -a
  from the help of ls : '' -a, --all do not ignore entries starting with.''
\item
  Add the options l to see rights on the files:
  ls -la
\item
  Get the help in git bash on windows:\\
  function --help
  example: ls --help
\item
  Make a file executable:\\
  chmod +x filename
\item
  Open a file with nano :\\
  nano my\_file.txt
\item
  Save changes made to a file in nano:\\
  ctrl+o + Enter + ctrl+x
\item
  `cd -' in order to come back to previous directory
\end{itemize}

  \bibliography{book.bib,packages.bib}

\end{document}
